\documentclass{article}

\begin{document}


\title{A Concept Paper on Bitcoin Mining Games}
\maketitle


\section{Introduction}
Bitcoin is known as the first decentralized digital currency in the world. Unlike any traditional currency issued and regulated by a sovereign bank, Bitcoin is not controlled by any institution or country. It circulates globally without boundary and is free from financial regulation systems due to its decentralized P2P accounting and transaction design. Bitcoin is a network protocol that enables individuals to transfer property rights on account units called "bitcoins", created in limited quantity. When an individual sends some bitcoins to another individual, this information is broadcast to the peer-to-peer Bitcoin network. However, for technical purposes we won’t address here, this transaction needs to be included – together with other transactions forming a block – in the blockchain in order to be confirmed and secured. As a consequence, the blockchain is a public ledger that contains the whole history of all the transactions of bitcoins ever processed and all Bitcoin users can trust this decentralized, distributed ledger.
It is the role of miners to do this work of confirming and securing transactions through insertion in the blockchain. Practically and slightly simplifying, for any miner, this work consists in considering a set of transactions that are present in the network, solving a mathematical problem that depends on this set and spreading the result to the Bitcoin network for this solution to be checked and for it to reach consensus. Once all these steps are done successfully, the set of transactions considered by the miner forms a block that is added on top of the blockchain. The first miner to succeed in this process is rewarded in bitcoins for his useful work.
In the current implementation of Bitcoin, this reward comes from both an ex-nihilo creation of some new bitcoins and some fees Bitcoin users can add to their transactions. Since some bitcoins are created in the mining process and in order to control the monetary base, mining is made more complex than it could be. And, since in a first approximation, the probability for each miner to solve a mining problem depends on his computational power, the complexity of mining is made proportional to the total computational power of all miners. More precisely, the complexity is dynamically adjusted so that a block solving – and hence a creation of bitcoins – occurs every ten minutes in expectation. Once a block is inserted in the blockchain, the mathematical problem faced by all miners are modified and we can consider that a fresh new speed game starts between miners. Hence, the whole building of the blockchain can be considered as independent block insertions from the miners’ point of view.

\section{Background}
Bitcoin mining Is a digital asset and a payment system invented by Satoshi Nakamoto. Nakamoto introduced the idea on 31 October 2008 to a cryptography mailing list and released it as an open source software in 2009.There have been several high profile claims to identity of Satoshi Nakamoto. However, none of them has provided proof beyond doubt that back up the claims. The system is peer to peer so transaction takes place directly, without an intermediary. These instructions are verified by network nodes and recorded in a public distributed ledger called the blockchain, which uses bitcoin as its unit of account. Since the system works without a central repository or single administrator, the US Treasury categorizes bitcoin as a decentralized virtual currency. It is often called the first cryptocurrency, although prior systems existed and it is described as decentralized digital currency. Bitcoin is the largest of its kind in terms of total market value. Bitcoins are created as a reward for payment processing work in which users offer their computing power to verify and record payments into a called public ledger. bitcoin mining has hardware which evolved dramatically since 2009.At first miners used central processing unit (CPU) to mine, but soon this wasn’t fast enough and it bogged down the system resources of the host computer. Miners quickly moved on to using graphical processing unit(GPU) in computer graphics cards because they were able to hash data 50 to 100 times faster and consumed much less power per unit of work. During the winter of 2011, a new industry sprung up with custom equipment that pushed the performance standards even higher. In the first wave of this specialty, bitcoin mining devices were easy to use. Bitcoin miners were based on field programmable gate array (FPGA) processors and attached to computer using a convenient method.

\section{Problem Statement}
Another application for bitcoins that is expected to become more important in the future is international payments, right now wiring money internationally involves slow, expensive and inconvenient services like western union. So this makes it very hard for miners to make payment within the stipulated period of time. And this has affected miners since it may not be affordable and neither is user friendly to them.

\section{Objectives}
\subsection{Main Objectives}
The main part of this research is to know how miners play the bitcoin mining games, the means of payment that is used and at what period of time one is expected to be paid.

\subsection{Specific Objectives}
To know what currency they use to pay for the game when one would like to play the game.
To make bitcoin mining games user friendly and affordable to players.

\section{Scope}
This research only focuses on bitcoin mining games in a way that it fully describes the activities that happen at bitcoin mining games hence giving a great deal of activities that is carried out there.

\section{Significance}
The purpose of this study enables us to know the overall activities done by bitcoin mining games since there is also calculating the hashing of data and how they are arranged discover the metrics and the means of currency used by bitcoin which is actually bitcoin.
This study will provide the individuals who are willing to join the bitcoin mining games the clear picture activities are done during the mining process.
This study will help the individuals who are willing to take part in the mining to learn how they are paid.

\section{Methodology}

We have used observation to view how bitcoin mining games are played using the internet.

We have used journals to inquire more about the bitcoin mining games.

We have used online tutorials to learn more about the bitcoin mining games.

We have visited different websites to learn more about bitcoin mining games.

\section{References}
bitcoin.org: Frequently asked questions. (https://bitcoin.org/en/faq) .

Coinbase.com. (https://www.coinbase.com) .

Exploring Miner Evolution in Bitcoin Network (http://wan.poly.edu/pam2015/papers/23.pdf) .

On the Instability of Bitcoin without the Block Reward (https://www.cs.princeton.edu/~smattw/CKWN-CCS16.pdf) .

\end{document}